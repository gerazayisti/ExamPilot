\documentclass[a4paper,12pt]{article}
\usepackage[utf8]{inputenc}
\usepackage[T1]{fontenc}
\usepackage[french]{babel}
\usepackage{graphicx}
\usepackage{hyperref}
\usepackage{geometry}
\usepackage{titlesec}
\usepackage{fancyhdr}
\usepackage{xcolor}

% Configuration de la page
\geometry{hmargin=2.5cm,vmargin=2.5cm}

% Couleurs personnalisées (Inspirées du thème Exam-Pilot)
\definecolor{primaryBlue}{RGB}{59, 130, 246} % #3b82f6
\definecolor{darkSlate}{RGB}{30, 41, 59}    % #1e293b

% Configuration des liens
\hypersetup{
    colorlinks=true,
    linkcolor=primaryBlue,
    filecolor=magenta,      
    urlcolor=cyan,
}

% En-tête et pied de page
\pagestyle{fancy}
\fancyhf{}
\rhead{\textcolor{gray}{Guide de l'Utilisateur}}
\lhead{\textbf{Exam-Pilot}}
\cfoot{\thepage}

% Titres
\titleformat{\section}
{\color{darkSlate}\normalfont\Large\bfseries}
{\thesection}{1em}{}

\titleformat{\subsection}
{\color{primaryBlue}\normalfont\large\bfseries}
{\thesubsection}{1em}{}

\title{
    \vspace*{2cm}
    \Huge \textbf{\textcolor{darkSlate}{Exam-Pilot}}\\
    \vspace{0.5cm}
    \large \textit{Le Pilote Automatique de vos Examens Universitaires}\\
    \vspace{2cm}
    \textbf{Guide de l'Utilisateur}\\
    \vspace{0.5cm}
    \small Version 1.0
}
\author{Équipe Exam-Pilot}
\date{\today}

\begin{document}

\maketitle
\thispagestyle{empty}
\newpage

\tableofcontents
\newpage

\section{Introduction}
Bienvenue sur \textbf{Exam-Pilot}. Ce logiciel a été conçu avec une mission simple : transformer le cauchemar logistique de la planification des examens en une tâche simple, rapide et sans erreur.

Exam-Pilot n'est pas un simple calendrier. C'est un assistant intelligent capable de :
\begin{itemize}
    \item Gérer vos salles, promotions et matières.
    \item Générer automatiquement un emploi du temps optimal sans conflits.
    \item Respecter les contraintes pédagogiques (espacement des examens, capacité des salles).
    \item Produire des documents officiels (PDF, Excel) prêts à être affichés.
\end{itemize}

Ce guide vous accompagnera dans la prise en main de l'outil, étape par étape.

\section{Premiers Pas}

\subsection{Accès et Connexion}
Pour accéder à Exam-Pilot, rendez-vous sur l'adresse fournie par votre administrateur (ou localhost en développement).
\begin{enumerate}
    \item **Inscription** : Lors de votre première visite, créez un compte en renseignant votre email, le nom de votre établissement et un mot de passe sécurisé.
    \item **Connexion** : Utilisez vos identifiants pour accéder à votre espace sécurisé.
\end{enumerate}

\subsection{Le Tableau de Bord}
Une fois connecté, vous arrivez sur le \textbf{Tableau de Bord} (Dashboard). C'est votre tour de contrôle. Vous y trouverez un résumé rapide de vos ressources : nombre de salles, de promotions enregistrées et la dernière session de planning générée.

\section{Gestion des Ressources}
Avant de pouvoir planifier des examens, vous devez dire à Exam-Pilot de quelles ressources vous disposez.

\subsection{Les Salles (Rooms)}
Allez dans l'onglet \textbf{Salles}. C'est ici que vous listez les espaces physiques disponibles.
\begin{itemize}
    \item Cliquez sur "Ajouter une salle".
    \item Renseignez le nom (ex: "Amphi 300"), la capacité (nombre de places) et le type.
    \item \textbf{Astuce} : Soyez précis sur la capacité. L'algorithme l'utilise pour savoir s'il peut placer un groupe d'étudiants dans cette salle.
\end{itemize}

\subsection{Les Promotions (Cohortes)}
L'onglet \textbf{Promotions} concerne vos étudiants. Une "Cohorte" est un groupe d'étudiants (ex: "Licence 1 - Informatique").
\begin{itemize}
    \item Créez une nouvelle cohorte en indiquant la filière, le niveau et l'effectif global.
    \item Une fois la cohorte créée, vous pourrez lui associer des \textbf{Unités d'Enseignement (UE)}. Ce sont les matières qui feront l'objet d'un examen.
\end{itemize}

\subsection{Import en Masse (Excel)}
Saisir les données élément par élément peut être fastidieux. Exam-Pilot propose une fonctionnalité d'importation rapide :
\begin{enumerate}
    \item Cliquez sur le bouton \textbf{"Importer"} dans l'onglet Salles ou Promotions.
    \item Téléchargez le \textbf{modèle Excel} fourni.
    \item Remplissez vos données (Nom, Capacité, ou Filière, Niveau, Effectif...) dans le fichier.
    \item Chargez le fichier rempli. Vos données seront ajoutées instantanément.
\end{enumerate}

\section{Génération du Planning Automatique}
C'est le cœur du système. Une fois vos salles et matières configurées, laissez la magie opérer.

\subsection{Lancer une Session}
Rendez-vous dans l'onglet \textbf{Planning}.
\begin{enumerate}
    \item Cliquez sur \textbf{"Nouvelle Session"}.
    \item Donnez un nom à votre session (ex: "Examens Semestre 1 - 2026").
    \item Choisissez la \textbf{date de début} et la \textbf{date de fin} de la période d'examens.
    \item Cliquez sur "Générer".
\end{enumerate}

\subsection{Ce que fait l'Algorithme}
La génération est ultra-rapide (environ 3 secondes chrono pour un planning standard). Pendant ce court instant, Exam-Pilot effectue des milliers de calculs pour :
\begin{itemize}
    \item \textbf{Éviter les conflits} : Une salle n'est jamais réservée deux fois en même temps.
    \item \textbf{Optimiser l'espace} : Il choisit la salle la plus adaptée à la taille de la promo.
    \item \textbf{Diversifier} : Il essaie de ne pas faire passer tous les examens d'une même filière le même jour, pour le confort des étudiants.
    \item \textbf{Respecter les distances} : Si vous avez activé l'option, il laisse un siège vide entre chaque étudiant (paramétrable).
\end{itemize}

\section{Visualisation et Export}

\subsection{Vue Tableau}
Une fois le planning généré, il s'affiche sous forme de tableau clair, trié par Date et Heure. Les informations identiques (Date, Heure, Filière) sont regroupées visuellement pour une lecture facile.

\subsection{Exports PDF et Excel}
Vous avez besoin d'imprimer le planning ou de l'envoyer par email ?
\begin{itemize}
    \item \textbf{Exporter en PDF} : Génère un document propre, formaté professionnellement avec en-têtes et colonnes parfaitement alignées, prêt à être affiché sur les murs de l'université.
    \item \textbf{Exporter en Excel} : Génère un fichier .xlsx modifiable si vous avez besoin de faire des ajustements manuels de dernière minute.
\end{itemize}

\section{Paramètres Avancés}
Dans l'onglet \textbf{Paramètres}, vous pouvez personnaliser l'expérience :
\begin{itemize}
    \item \textbf{Identité} : Changez le nom de l'établissement et la couleur principale de l'interface.
    \item \textbf{Anti-triche} : Configurez l'espace entre les étudiants ("Gap"). Par exemple, réglez sur "1" pour laisser une chaise vide entre chaque étudiant lors du calcul de capacité des salles.
\end{itemize}

\vspace{2cm}
\begin{center}
    \textit{Exam-Pilot : Laissez l'informatique gérer la complexité, concentrez-vous sur la pédagogie.}
\end{center}

\end{document}
